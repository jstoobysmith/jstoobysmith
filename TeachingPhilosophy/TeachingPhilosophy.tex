\documentclass[12pt,letter]{article}
%%%
%Bool
\newcommand{\includePublications}{T} %T for true.
%%%
\usepackage[margin=0.5in]{geometry}
\usepackage{hyperref}
\renewcommand{\arraystretch}{1.5} 
\usepackage{enumitem}
\usepackage{xcolor}
%\definecolor{mycolor}{RGB}{37,129,120}
%\definecolor{mycolor}{RGB}{128, 0, 32}
\definecolor{mycolor}{RGB}{0, 128, 128}
%Comment
\newcommand{\js}[1]{{\leavevmode\color{magenta}\bf  JS: #1}}
%%%%%%%%%%%%%
%Title and entry
\usepackage{colortbl}
\newcommand{\mytitle}[1]{
   \setlength\arrayrulewidth{1.5pt}
	\noindent\begin{tabular}{p{0.9\textwidth}p{0.05\textwidth}}
	{\Large\bf #1}&{\hfill \large }\\
	\arrayrulecolor{mycolor}\hline
	\end{tabular}
}
\pagestyle{empty} 

\newcommand{\myentry}[2]{
	\noindent\begin{tabular}{p{0.1\textwidth}p{0.85\textwidth}}
	\textcolor{mycolor}{#1}& #2
	\end{tabular}\\
}
\usepackage{textcomp}
\newcommand{\mybullet}{\textcolor{mycolor}{$\ast$}\ }
%%%%%
\usepackage{titlesec}
\titleclass{\customtitle}{straight}[\section]
\newcounter{customtitle}
\titleformat{\customtitle}[block]
  {\normalfont}
  {}
  {0em}
  {\mytitle}
\titlespacing*{\customtitle}{-0.2cm}{*0.5}{0.1cm}
%%%%%%%%%%%%%
%For colored bar
\usepackage{eso-pic}


\AddToShipoutPictureBG*{%
 \AtPageUpperLeft{%
 \textcolor{mycolor}{\rule[-2.6cm]{\paperwidth}{4cm}}}}


\AddToShipoutPictureBG{%
 \AtPageUpperLeft{%
 \textcolor{mycolor}{\rule[-1cm]{\paperwidth}{4cm}}}}


\AddToShipoutPictureBG{%
 \AtPageLowerLeft{%
 \textcolor{mycolor}{\rule[-1.2cm]{\paperwidth}{2cm}}}}
 
\usepackage{tabularx}
%%%%%%%%%%%%%
%\usepackage{draftwatermark}
%\SetWatermarkText{\color{mycolor} DRAFT}
%\SetWatermarkScale{9}

\begin{document}

%\center{(See separate document for publication list)}
\noindent{{\Huge \color{white}\bf Teaching philosophy}\newline  }
\vspace{-1cm}
\begin{flushright}
{{\Large \color{white}Joseph Tooby-Smith}}
\end{flushright}
\vspace{0.4cm}

My teaching philosophy is based on five principals, which I will discuss here in turn. This discussion also includes details of my previous teaching experience. Following this, I will discuss my commitment to diversity, equality, and inclusivity (DEI) in the classroom and academia more broadly, my teaching plans for the 
academic year 2024-2025, and proposed modules.
 
%%%%%%%%%  
\customtitle{1. Tailored, level-appropriate material}
 %%%%%%%%%  
 
 To ensure that the transfer of knowledge is effectively passed on to an audience, I believe that material should be presented at an appropriate knowledge-based level. 
 
 My research spans the boundaries of physics, mathematics, and computer science. Often, I present material from a discipline to academics who work in a different discipline. In these instances, I make a concerted effort to understand the audience's background and tailor my presentation accordingly. For example, when talking about category theory to physicists, I ensure to provide an introduction to basic concepts before delving into more complex topics, something which may not be necessary with certain mathematicians or computer scientists.


%%%%%%%%%  
\customtitle{2. Actively listening to feedback}
 %%%%%%%%%  
 To become a better and more effective teacher, it is crucial to actively seek and act on explicit and implicit feedback.
 
 During my PhD, I served as a teaching assistant (TA) for several courses, where I guided students through answers to problem sets. In one course (Gauge Field Theory), a formal feedback mechanisms was lacking, so I created an anonymous Google survey to gather student feedback. As an example, one student's suggestion was to use an overhead projector in-addition of the blackboard for presenting solutions.  This was promptly implemented, improving the clarity of my teaching.
 
 During my time as an undergraduate, I took an elective course on `Teaching in Schools'. The main focus of this course was on spotting and addressing misconceptions in the classroom.  
 During this course, I gave a group of students short multiple-choice questionnaires designed specifically to detect misconceptions the students may have regarding Newton's laws and forces. This provided a valuable implicit form of feedback to the teacher. 
 
 %%%%%%%%%  
\customtitle{3. Student engagement through active learning}
 %%%%%%%%%
To deepen comprehension and maintain student engagement, I believe in employing active learning techniques, especially in small classroom settings. Active learning not only keeps students engaged but also helps identify areas where they may be struggling, providing an implicit form of feedback (see above).

For example, as a TA during my PhD, I encouraged students to explain concepts using the blackboard, helping  them understand the material better, as well as helping me identify the areas they struggled with. 

During my undergraduate at Oxford I was heavily involved in organising an outreach program called Oxford Hands-On-Science. This involved going to schools in socio-economically  disadvantaged areas to show and teach students about science. During these outreach events we got the students actively involved with demonstrations. For example, when demonstrating the digestive system,  I had each student replicate a different part of the digestive system using household items, ensuring every student participated actively.

  
\customtitle{4. Fostering a sense of accomplishment}
 %%%%%%%%%  
To cultivate a love of learning, I believe that every student should feel a sense of accomplishment from every course they take. Even if they do not excel in formal assessments, they should not regret having taken the course, and leave with a positive experience.


In my future teaching, I plan to achieve this by incorporating small research projects into assignments or long-form coursework, designed at an appropriate level. For example, I discuss in my research plan how I intend to use a project I worked on called HepLean to engage students in active research.

  %%%%%%%%%  
\customtitle{5. Varied presentation styles}
 %%%%%%%%%  
 As someone with profound dyslexia, I understand the importance of diverse presentation styles to accommodate different learning needs. I believe it is essential to present information in multiple formats to ensure that all students can engage with the material effectively.
 
 For example, as part of the demonstrations with Oxford Hands-On-Science (discussed above), I would start each demonstration with a discussion to assess prior knowledge, followed by a visual demonstration and then a hands-on experiment. This repetition and variation in presentation helped reinforce the concepts for all students.
   
%%%%%%%%%  
\customtitle{Diversity, equality and inclusivity}
%%%%%%%%% 
To ensure that academia and the universities are fair, inclusive, and welcoming places, I have a commitment to promoting diversity, equality, and inclusivity. Two specific examples illustrate this commitment.

The first example is through my involvement with Oxford Hands-On-Science where I visited high schools in socio-economically disadvantaged parts of the UK to inspire students who might not otherwise have access to scientific role models. Our goal was to spark an interest in science and technology among these students.

The second example is my work in improving accessibility in academia. 
 As someone with profound dyslexia, I am acutely aware of the challenges faced by neurodiverse students and academics. To improve accessibility, I have created short videos to accompany some of my academic papers and have actively engaged with the team behind arXiv to discuss accessibility issues in research. For example, I was interviewed as part of the research that led to specific accessibility improvements.

In the classroom, I believe it is vital to establish clear expectations at the start of the course regarding student participation and my role as an instructor. Moreover, during active learning sessions, I aim to ensure that discussions are not dominated by a small subset of students by setting clear ground rules that promote inclusivity.

%%%%%%%%%  
\customtitle{Teaching plan for 2024-2025}
%%%%%%%%% 
In the academic year 2024-2025 to enhance my teaching, and get more practice teaching within computer science, I will help TA for a course based on Agda and type theory, and a course on Algebra and combinatorics. Both these courses are led by Tarmo Uustalu. I will sit in a number of computer science based courses by other lectures to experience different teaching styles within computer science.

As part of my professional development I will apply for an advanced HE associate fellowship.

 %%%%%%%%%  
\customtitle{Proposed modules}
%%%%%%%%% 

I'm particular excited to teach  and develop modules on functional programming,  type theory and category theory in computer science,  proof assistants in CS and mathematics and scientific computing.  
%%%%%%%%%
%\begingroup
%\renewcommand{\section}[2]{}%
%%\renewcommand{\chapter}[2]{}% for other classes
%\begin{thebibliography}{}
%\bibitem{Davighi_Gripaios_Tooby-Smith_2020}
%    J. Davighi, B. Gripaios, and J. Tooby-Smith, J. Phys. A: Math. Theor. 53, 145302 (2020).
%\bibitem{U1Anomaly}
%    B. C. Allanach, B. Gripaios, and J. Tooby-Smith, J. High Energ. Phys. 2020, 65 (2020).
%\bibitem{GaugeRank}
%B. C. Allanach, B. Gripaios, and J. Tooby-Smith, Phys. Rev. D 101, 075015 (2020).
%\bibitem{ExtraGaugeBoson}
%B. C. Allanach, B. Gripaios, and J. Tooby-Smith, ArXiv:2006.03588 (2020).
%\bibitem{SuperSoft}
%T. Cohen, N. Craig, S. Koren, M. McCullough, and J. Tooby-Smith, ArXiv:2002.12630 (2020).
%\end{thebibliography}
%\endgroup

%%%%%%%%%%%%%%
\end{document}
