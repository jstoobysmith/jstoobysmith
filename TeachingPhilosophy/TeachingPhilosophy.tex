\documentclass[12pt,letter]{article}
%%%
%Bool
\newcommand{\includePublications}{T} %T for true.
%%%
\usepackage[margin=0.5in]{geometry}
\usepackage{hyperref}
\renewcommand{\arraystretch}{1.5} 
\usepackage{enumitem}
\usepackage{xcolor}
%\definecolor{mycolor}{RGB}{37,129,120}
%\definecolor{mycolor}{RGB}{128, 0, 32}
\definecolor{mycolor}{RGB}{0, 128, 128}
%Comment
\newcommand{\js}[1]{{\leavevmode\color{magenta}\bf  JS: #1}}
%%%%%%%%%%%%%
%Title and entry
\usepackage{colortbl}
\newcommand{\mytitle}[1]{
   \setlength\arrayrulewidth{1.5pt}
	\noindent\begin{tabular}{p{0.9\textwidth}p{0.05\textwidth}}
	{\Large\bf #1}&{\hfill \large }\\
	\arrayrulecolor{mycolor}\hline
	\end{tabular}
}
\pagestyle{empty} 

\newcommand{\myentry}[2]{
	\noindent\begin{tabular}{p{0.1\textwidth}p{0.85\textwidth}}
	\textcolor{mycolor}{#1}& #2
	\end{tabular}\\
}
\usepackage{textcomp}
\newcommand{\mybullet}{\textcolor{mycolor}{$\ast$}\ }
%%%%%
\usepackage{titlesec}
\titleclass{\customtitle}{straight}[\section]
\newcounter{customtitle}
\titleformat{\customtitle}[block]
  {\normalfont}
  {}
  {0em}
  {\mytitle}
\titlespacing*{\customtitle}{-0.2cm}{*0.5}{0.1cm}
%%%%%%%%%%%%%
%For colored bar
\usepackage{eso-pic}


\AddToShipoutPictureBG*{%
 \AtPageUpperLeft{%
 \textcolor{mycolor}{\rule[-2.6cm]{\paperwidth}{4cm}}}}


\AddToShipoutPictureBG{%
 \AtPageUpperLeft{%
 \textcolor{mycolor}{\rule[-1cm]{\paperwidth}{4cm}}}}


\AddToShipoutPictureBG{%
 \AtPageLowerLeft{%
 \textcolor{mycolor}{\rule[-1.2cm]{\paperwidth}{2cm}}}}
 
\usepackage{tabularx}
%%%%%%%%%%%%%
\usepackage{draftwatermark}
\SetWatermarkText{\color{mycolor} DRAFT}
\SetWatermarkScale{9}

\begin{document}

%\center{(See separate document for publication list)}
\noindent{{\Huge \color{white}\bf Teaching philosophy}\newline  }
\vspace{-1cm}
\begin{flushright}
{{\Large \color{white}Joseph Tooby-Smith}}
\end{flushright}
\vspace{0.4cm}

My teaching philosophy is based on five principals, which I will discuss in turn here. This will is followed by a discussion of my commitment to diversity, equality, and inclusivity in the classroom and acdemia in general.
 
%%%%%%%%%  
\customtitle{Tailored, level-appropriate material}
 %%%%%%%%%  
 
 To ensure that the transfer of knowledge is effectively passed on to an audience, I believe that material should be presented in a level and knowledge appropriate way. 
 
 For example, during undergraduate at Oxford I was heavily involved in organising an outreach program called Oxford Hands-On-Science. This involved going to schools in scoio-econically disadvantaged areas to show and teach students about science. The demonstrates that we undertook were specifically designed for the ages of the students we were talking too. 
 
 Another example, is from my presentation of academic research in seminars. My work is at the boundary of different disciplines (physics, mathematics and computer science), and on many occasions I have presented material from one  discipline to academics in others. When doing this I have been careful to understand the background of the audience and what I can expect them know and not know. For example, talking about category theory to physicists requires an introduction to what a category is etc.
 
 
  Having worked at the boundary of 
three different disciplines (physics, mathematics and computer science), and having 
presented on many occasions material from one  discipline to academics in others, 
I am aware of the importance of presenting material at the correct level for the audience.
I'm also aware of what goes wrong when this is not done. 


%%%%%%%%%  
\customtitle{Actively listening to feedback}
 %%%%%%%%%  
 To ensure that one becomes a better and more effective teacher, I believe it is important to both actively seek feedback, and to act on the feedback given. 
 
 During my PhD I was a teaching assistent for a number of courses. This involved working through solutions to problems sets with students. For one of the courses I TAed for (Gauge field theory) there was formal method for the students to give feedback. I thus created an anomonous google survey for the students to fill out at-their-will. One students suggested that the answers be presented on the over-head projector, rather than just on the blackboard (as I was doing). I followed this feedback in future classes.
 
  
  \js{Teaching in schools. }
 %%%%%%%%%  
\customtitle{Student engagment through active learning}
 %%%%%%%%%
 To deepen comprehension of the material, and to keep the students actively   engaged with the material, I believe that active learning techniques should applied, especially in small-classroom
settings. Such active learning, when correctly observed, has the additional benefit of help identify areas which the students 
do not understand, and thus provide an implicit form of feedback.

One such example of active learning I have implemented when TAing as a PhD student is getting the students to explain the material at the board themselves. 

Another example, which I used when doing outreach secessions, is provide a demonstration in which all students have part. For example, one of the demonstrations for younger students was about the digestive system. Here each student was given part of the  digestive system to replicate with various household items.  

  
\customtitle{Fostering a sense of accomplishment:}
 %%%%%%%%%  
 To foster a love of learning, 
  I believe that every student should
  feel an accomplishment from every course they take. Even if they do not do particularly well in formal assessments, they should leave a course with no regrets that they took it. 
  
  One way I plan on doing this in my future teaching is to  engage the students n small bits of active research, through either 
home-work assignments or more long-form course work. This should, of course, 
be designed at a level appropriate level. In my research plan I discuss how I plan to do this using a project I worked on called HepLean. 

  %%%%%%%%%  
\customtitle{Varied presentation styles}
 %%%%%%%%%  
 
 As someone with dyslexia, I understand
 the consequences of sitting in lectures  where the presentation 
style did not meet my needs. Thus I believe it is important that a varied presentation 
style is used for each and every bit of information given to the students. 
This sometimes will involve retracing material more than once, however this 
is not necessary a bad thing, as it can be used as reinforcement opportunities. 

 
%%%%%%%%%  
\customtitle{DEI}
%%%%%%%%% 
To ensure that academic, and the universities is a fair, inclusive and welcoming place, I have a commitment promoting diversity, equality and inclusivity. Let me discuss two things I have done promote DEI. 

Firstly, as mentioned above I was heaivly involved in Oxford Hands-on-Science. As part of this we visited high-schools from scoio-econically disadvantaged areas in the UK to interact with students who would not normally get to speak to scientists. Part of the aim was students from these areas inspired to study and learn more about the sciences and technology. 

Secondly, with profound dyslexia I have always stuggled with some aspects of adademia. This includes, for example reading and writing papers. Thus aware of the difficulties neurodiverse students and ademics can have accessing information from papers, I have made short videos for some my of my academic papers. I have also actively engaged with the team behind arXiv about accessibility in research. For example, I was interviewed  as part of the reasearch that lead to \js{...}. 

To promote DEI in the classroom I believe it is particular important a set out explications at the start of the course of what I expect from the students and what they should expect from me. I also believe it is important that when doing active learning (as discussed above), the conversation is not dominated by a small subset of the students. This is something which can be prevented with clear ground rules.  



%%%%%%%%%
%\begingroup
%\renewcommand{\section}[2]{}%
%%\renewcommand{\chapter}[2]{}% for other classes
%\begin{thebibliography}{}
%\bibitem{Davighi_Gripaios_Tooby-Smith_2020}
%    J. Davighi, B. Gripaios, and J. Tooby-Smith, J. Phys. A: Math. Theor. 53, 145302 (2020).
%\bibitem{U1Anomaly}
%    B. C. Allanach, B. Gripaios, and J. Tooby-Smith, J. High Energ. Phys. 2020, 65 (2020).
%\bibitem{GaugeRank}
%B. C. Allanach, B. Gripaios, and J. Tooby-Smith, Phys. Rev. D 101, 075015 (2020).
%\bibitem{ExtraGaugeBoson}
%B. C. Allanach, B. Gripaios, and J. Tooby-Smith, ArXiv:2006.03588 (2020).
%\bibitem{SuperSoft}
%T. Cohen, N. Craig, S. Koren, M. McCullough, and J. Tooby-Smith, ArXiv:2002.12630 (2020).
%\end{thebibliography}
%\endgroup

%%%%%%%%%%%%%%
\end{document}
