\documentclass[12pt,letter]{article}
%%%
%Bool
\newcommand{\includePublications}{T} %T for true.
%%%
\usepackage[margin=0.5in]{geometry}
\usepackage{hyperref}
\renewcommand{\arraystretch}{1.5} 
\usepackage{enumitem}
\usepackage{xcolor}
%\definecolor{mycolor}{RGB}{37,129,120}
%\definecolor{mycolor}{RGB}{128, 0, 32}
\definecolor{mycolor}{RGB}{0, 128, 128}
%%%%%%%%%%%%%
%Title and entry
\usepackage{colortbl}
\newcommand{\mytitle}[1]{
   \setlength\arrayrulewidth{1.5pt}
	\noindent\begin{tabular}{p{0.9\textwidth}p{0.05\textwidth}}
	{\Large\bf #1}&{\hfill \large }\\
	\arrayrulecolor{mycolor}\hline
	\end{tabular}
}
\pagestyle{empty} 

\newcommand{\myentry}[2]{
	\noindent\begin{tabular}{p{0.1\textwidth}p{0.85\textwidth}}
	\textcolor{mycolor}{#1}& #2
	\end{tabular}\\
}
\usepackage{textcomp}
\newcommand{\mybullet}{\textcolor{mycolor}{$\ast$}\ }
%%%%%
\usepackage{titlesec}
\titleclass{\customtitle}{straight}[\section]
\newcounter{customtitle}
\titleformat{\customtitle}[block]
  {\normalfont}
  {}
  {0em}
  {\mytitle}
\titlespacing*{\customtitle}{-0.2cm}{*0.5}{0.1cm}
%%%%%%%%%%%%%
%For colored bar
\usepackage{eso-pic}


\AddToShipoutPictureBG*{%
 \AtPageUpperLeft{%
 \textcolor{mycolor}{\rule[-2.6cm]{\paperwidth}{4cm}}}}


\AddToShipoutPictureBG{%
 \AtPageUpperLeft{%
 \textcolor{mycolor}{\rule[-1cm]{\paperwidth}{4cm}}}}


\AddToShipoutPictureBG{%
 \AtPageLowerLeft{%
 \textcolor{mycolor}{\rule[-1.2cm]{\paperwidth}{2cm}}}}
 
\usepackage{tabularx}
%%%%%%%%%%%%%
%\usepackage{draftwatermark}
%\SetWatermarkText{\color{mycolor} DRAFT}
%\SetWatermarkScale{9}

%% Comments 

\begin{document}

%\center{(See separate document for publication list)}
\noindent{{\Huge \color{white}\bf Research Plan}\newline  }
\vspace{-1cm}
\begin{flushright}
{{\Large \color{white}Joseph Tooby-Smith }}
\end{flushright}
\vspace{0.4cm}


%%%%%%%%%%  
\customtitle{The past}
%%%%%%%%%%  

I started my academic career with a PhD and a postdoc in theoretical physics. During my time as a physicist, my speciality involved applying techniques from pure mathematics to solve problems in physics. 

One area of mathematics that I used extensively was category theory. Category theory is seen by some as a unifying language for mathematics. My work showed how it can be used as a tool for interpretation and as a way to derive new results in physics. This is very similar to its use in computer science and specifically functional programming languages. I went on to specialise in an area of higher category theory called higher topos theory. Using this I derived new results related to an area of physics of current interest, called generalised symmetries. I'm proud of this work because it demonstrates an explicit use of higher category theory in deriving new results.

My work in higher topos theory resulted in me learning about homotopy type theory (the two have a subtle relationship). Homotopy type theory, itself, then resulted in me learning about theorem proving languages such as coq and Lean. I came to understand the potential benefit that such programs could have to the physical sciences. Thus, I made it my goal to promote the use of interactive theorem provers in the physical sciences, and overcome the hurdles prohibiting their use. This work, which has been self-directed and independent of faculty-involvement, is what I would like to be known for in the future.

In pursuit of this goal, my research turned from physics to computer science. My research turned into the use of underlying structures in functional programming, and applying them directly or indirectly to make interactive theorem provers easier for physical scientists. Solidifying this transition for my second postdoc, I joined a computer science department at Reykjavik University working with Tarmo Uustalu.

My most important work in this area to date, is the development of a library called HepLean (for which there is a corresponding preprint in the latter stages of review). This is similar to the project Mathlib and aims to formalise results from high energy physics in Lean 4. The project HepLean makes clear the benefits the physical scientist can gain from using a well-structured library written in an interactive theorem prover. Let me briefly mention some of these:
\begin{itemize}
\item It stores information from the area in a linear fashion, making look-up easier.
 \item It opens the door for the automatic derivation of new results in�
the physical sciences using AI and other automated tactics for theorem proving. 
\item It allows the community to automatically review papers and results for mathematical�correctness. Mathematical correctness is not something usually reviewed for in some areas of physics. 
\item  It opens the door for new pedagogical methods in�both computer science and the physical sciences by for example using Lean games. 
\end{itemize}

%%%%%%%%%%  
\customtitle{Current and future work in interactive theorem proving}
%%%%%%%%%%  

Let me turn to current and future work. I have recently had preprint out on the formalization of index notation into Lean 4 (arXiv: 2411.07667). The aim of this is to overcome one the hurdles 
prohibiting the use of interactive theorem provers in the physical sciences. Index notation allows physicist to write tensors and operations between them in a concise way. It's formalization is a balancing act between making it easy for the physicist to use but formal so that Lean accepts it, and general enough to cover the different types of index notation used by physicists. The way that I achieved this is through a novel application of category theory, based roughly on the theory of operads.  
Based on this implementation, part of my current work is developing tactics in Lean to make it easier to prove results related to tensors and index notation. 

There are other hurdles similar to the notation one just discussed, that prohibit the use of interactive theorem provers in the physical sciences. One important hurdle is the learning curve that physicists must undergo to learn Lean and the current lack of reward they get from doing so. As a way to reduce the learning curve I plan to utilise informal definitions and lemmas in Lean. These are `english' written results which can be later formalised by experts in Lean (or even better by AI), but are much easier for the physicist to write. There is a crude implementation of this already in HepLean. In addition I plan to learn how AI can be used to formalise the physical sciences and so increasing part of the reward for physicists. There is already a lot of work on using AI to formalise mathematics (e.g., DeepMind's work on Math Olympiad problems), which may not transition to the physical sciences smoothly, due to differing conventions between the communities. 

%%%%%%%%%%  
\customtitle{Future work in category theory and computer science}
%%%%%%%%%%  
As part of my future work, I plan to explore the intersection of category theory and computer science. I am specifically interested in the role higher category theory and higher topos theory can play in computer science. A key reference for this work will be Jacob Lurie's book on higher algebra, the material of which I am familiar with from my use of category theory in physics. In this direction, I am currently looking into higher-categorical generalizations of directed containers (Ahman, Chapman \& Uustalu). 

 
\customtitle{Plan for student involvement} 


HepLean offers numerous opportunities for involving students in research. I plan to develop three lists of undergraduate-level  and graduate-level projects around HepLean:

\begin{itemize}
	\item Functional programming projects: An example of such a project would be the handling of lists in Lean to efficiently undertake computations needed for index notation (a notation used by physicists to deal with tensors). Other examples will involve meta-programming in Lean to make the user-experience easier.
\item AI in Physics and Mathematics: These projects will explore auto-formalization of theorems in physics (converting human-written results into Lean proofs) and the inverse process, "auto-informalization." While these techniques have been explored in mathematics, they remain largely unexplored in the physical sciences, as dicussed above.
\item Interdisciplinary Theorem Proving: These projects will involve proving physics theorems using Lean. Many such problems require minimal prerequisites in physics once the theorem is stated. For instance, formalizing properties of the two-Higgs doublet model potential could be an excellent project for students. Physicists are interested in its properties, 
such as its minima, whether it is bounded or not etc.
\end{itemize}
Each of these project lists will include homework-style tasks that can be completed in a few hours and more detailed thesis-level projects. The breadth of HepLean makes it relatively easy to generate a diverse range of projects.

%
%
%\noindent My research lies at the intersection of computer science, mathematics, and physics. I am particularly interested in building bridges between these disciplines using interactive theorem proving. I hold a PhD in theoretical physics from the University of Cambridge, and have completed a postdoctoral fellowship at Cornell University, where I focused on applying theorem proving software to physics. In the academic year 2024-2025, I am undertaking a postdoctoral position in computer science at the University of Reykjavik.
%%%%%%%%%%  
%\customtitle{The past}
%%%%%%%%%%  
%
%A significant part of my past research has involved digitizing results from high-energy physics into a Lean 4 library called "HepLean." Lean 4 is an interactive theorem prover that automatically verifies the correctness of proofs. The main motivations behind this project include:
%\begin{enumerate}
%\item A linear-storage of information: Currently information in high energy physics is stored non-linearly, with material related to the same subject spread across papers indexed only by an arXiv number at best. In HepLean results are stored linearly, with results related to the same area of high energy physics being stored together, making the look-up of information more efficient.
%\item Automated methods to derive new results:  In Lean, one can write "tactics" to attempt automatic completion of proofs. There is also ongoing work on using AI to prove theorems in Lean, with such proofs being automatically certified as correct. Notably, Google DeepMind has explored this in the context of the Math Olympiad.
%\item Automatic review of results for mathematical correctness: HepLean provides a method for authors and reviewers to be confident of the mathematical correctness of results. In this regard, Lean's automatic proof verification ensures the rigour and accuracy of results.
%\item New pedagogy methods:	 HepLean offers a novel approach to teaching both functional programming and physics, allowing students to engage in active research and learn through direct interaction with theorem proving software.
%\end{enumerate}
% This is the first such project in high energy physics. However, there is a similar project for mathematics called Mathlib, which forms the mathematical foundation underlying HepLean.
%
%
%%%%%%%%%%  
%\customtitle{The future}
%%%%%%%%%%  
%My future research will continue the development of HepLean, with the aim of making it more accessible to physical scientists.  
%To achieve this overarching goal, I plan to undertake the following specific research steps:
%\begin{enumerate}
%\item Blueprint:  Lean 4, a "blueprint" is an English-written document that outlines all the steps needed to convert an English-written proof into a Lean proof. I plan to create such blueprints for various theories in physics, which will serve as pseudo-code, bridging the gap between traditional physics work and interactive theorem provers.
%\item Artificial intelligence: While AI has been applied to mathematical problems in Lean (e.g., DeepMind's work on Math Olympiad problems), I aim to extend this to physics. I plan to create a dataset of Lean 4-written theorems from physics that can be used for AI testing and training.
%\item Tensors: High-energy physics relies heavily on tensors. I plan to develop Lean 4 tactics that can formally verify results related to tensors, which will be crucial for ensuring the correctness of complex calculations in physics.
%\end{enumerate}
%
%There will be also be a focus on HepLeans theoretical and computer science foundations. This will involve applying techniques from functional programming, AI, and category theory to create foundational definitions and functions to ease future development.
%
%\customtitle{Plan for undergraduate student involvement} 
%
%HepLean offers numerous opportunities for involving undergraduate students in research. I plan to develop three lists of undergraduate-level projects around HepLean:
%
%\begin{itemize}
%	\item Functional programming projects: An example of such a project would be the handling of lists in Lean to efficiently undertake computations needed for index notation (a notation used by physicists to deal with tensors). Other examples will involve meta-programming in Lean to make the user-experience easier.
%\item AI in Physics and Mathematics: These projects will explore auto-formalization of theorems in physics (converting human-written results into Lean proofs) and the inverse process, "auto-informalization." While these techniques have been explored in mathematics, they remain largely unexplored in the physical sciences, as dicussed above.
%\item Interdisciplinary Theorem Proving: These projects will involve proving physics theorems using Lean. Many such problems require minimal prerequisites in physics once the theorem is stated. For instance, formalizing properties of the two-Higgs doublet model potential could be an excellent project for students. Physicists are interested in its properties, 
%such as its minima, whether it is bounded or not etc.
%\end{itemize}
%Each of these project lists will include homework-style tasks that can be completed in a few hours and more detailed thesis-level projects. The breadth of HepLean makes it relatively easy to generate a diverse range of projects.
%
%
%%%%%%%%%%  
%\customtitle{Other research}
%%%%%%%%%%  
%In addition to my work on HepLean, I have a strong background in applying category theory outside of pure mathematics. Historically, I have used category theory as a language to reframe problems in the physical sciences and derive new results. For example, in high-energy physics, there is a relatively new concept called ``generalized symmetry'' In \js{} we used higher topoi - a special type of higher category - to define and derive new results about these symmetries. Higher topos theory is actually related to homotopy type theory, which led me to interactive theorem proving.
%
%Category theory is also heavily used in functional programming. For instance, the concept of a monad in functional programming is a categorical definition. A monad is a special case of a more general object in the theory of higher algebra. 
%This is an area I have expertise in, since it overlaps with my use of category theory in physics. 
%
%
%In the future, I plan to investigate the role that higher category theory can play in functional programming and, more specifically, in interactive theorem provers for the physical sciences. Currently, I am exploring the theory of modular operads to develop an efficient method for index notation of tensors (mentioned briefly above).
%%%%%%%%%
%\begingroup
%\renewcommand{\section}[2]{}%
%%\renewcommand{\chapter}[2]{}% for other classes
%\begin{thebibliography}{}
%\bibitem{Davighi_Gripaios_Tooby-Smith_2020}
%    J. Davighi, B. Gripaios, and J. Tooby-Smith, J. Phys. A: Math. Theor. 53, 145302 (2020).
%\bibitem{U1Anomaly}
%    B. C. Allanach, B. Gripaios, and J. Tooby-Smith, J. High Energ. Phys. 2020, 65 (2020).
%\bibitem{GaugeRank}
%B. C. Allanach, B. Gripaios, and J. Tooby-Smith, Phys. Rev. D 101, 075015 (2020).
%\bibitem{ExtraGaugeBoson}
%B. C. Allanach, B. Gripaios, and J. Tooby-Smith, ArXiv:2006.03588 (2020).
%\bibitem{SuperSoft}
%T. Cohen, N. Craig, S. Koren, M. McCullough, and J. Tooby-Smith, ArXiv:2002.12630 (2020).
%\end{thebibliography}
%\endgroup

%%%%%%%%%%%%%%
\end{document}
