\documentclass[14pt,letter]{article}
%%%
%Bool
\newcommand{\includePublications}{T} %T for true.
%%%
\usepackage[margin=0.5in]{geometry}
\usepackage{hyperref}
\renewcommand{\arraystretch}{1.5} 
\usepackage{enumitem}
\usepackage{xcolor}
%\definecolor{mycolor}{RGB}{37,129,120}
%\definecolor{mycolor}{RGB}{128, 0, 32}
\definecolor{mycolor}{RGB}{0, 128, 128}
%Comment
\newcommand{\js}[1]{{\leavevmode\color{magenta}\bf  JS: #1}}
%%%%%%%%%%%%%
%Title and entry
\usepackage{colortbl}
\newcommand{\mytitle}[1]{
   \setlength\arrayrulewidth{1.5pt}
	\noindent\begin{tabular}{p{0.9\textwidth}p{0.05\textwidth}}
	{\Large\bf #1}&{\hfill \large }\\
	\arrayrulecolor{mycolor}\hline
	\end{tabular}
}
\pagestyle{empty} 

\newcommand{\myentry}[2]{
	\noindent\begin{tabular}{p{0.1\textwidth}p{0.85\textwidth}}
	\textcolor{mycolor}{#1}& #2
	\end{tabular}\\
}
\usepackage{textcomp}
\newcommand{\mybullet}{\textcolor{mycolor}{$\ast$}\ }
%%%%%
\usepackage{titlesec}
\titleclass{\customtitle}{straight}[\section]
\newcounter{customtitle}
\titleformat{\customtitle}[block]
  {\normalfont}
  {}
  {0em}
  {\mytitle}
\titlespacing*{\customtitle}{-0.2cm}{*0.5}{0.1cm}
%%%%%%%%%%%%%
%For colored bar
\usepackage{eso-pic}


\AddToShipoutPictureBG*{%
 \AtPageUpperLeft{%
 \textcolor{mycolor}{\rule[-2.6cm]{\paperwidth}{4cm}}}}


\AddToShipoutPictureBG{%
 \AtPageUpperLeft{%
 \textcolor{mycolor}{\rule[-1cm]{\paperwidth}{4cm}}}}


\AddToShipoutPictureBG{%
 \AtPageLowerLeft{%
 \textcolor{mycolor}{\rule[-1.2cm]{\paperwidth}{2cm}}}}
 
\usepackage{tabularx}
%%%%%%%%%%%%%
\usepackage{draftwatermark}
\SetWatermarkText{\color{mycolor} DRAFT}
\SetWatermarkScale{9}

\begin{document}

%\center{(See separate document for publication list)}
\noindent{{\Huge \color{white}\bf Teaching philosophy}\newline  }
\vspace{-1cm}
\begin{flushright}
{{\Large \color{white}Joseph Tooby-Smith}}
\end{flushright}
\vspace{0.4cm}

My teaching philosophy is based on five principals, which I will discuss in turn here. This will is followed by a discussion of my commitment to diversity, equality, and inclusivity in the classroom and acdemia in general.
 
%%%%%%%%%  
\customtitle{Tailored, level-appropriate material}
 %%%%%%%%%  
  Having worked at the boundary of 
three different disciplines (physics, mathematics and computer science), and having 
presented on many occasions material from one  discipline to academics in others, 
I am aware of the importance of presenting material at the correct level for the audience.
I'm also aware of what goes wrong when this is not done. 

\js{Phd students, vs, schools, vs undergrad}

%%%%%%%%%  
\customtitle{Actively listening to feedback}
 %%%%%%%%%  
  \js{Anomymous survays, acting on the feedback.}
  
 %%%%%%%%%  
\customtitle{Student engagment through active learning}
 %%%%%%%%%  
  I believe, especially in small-classroom,
settings that student engagement should be a priority. This has dual benefits. 
Firstly,for example, getting students to explain material can positive impact
on their learning, depending comprehension. It also provides them a different channel through which 
they can learn. Secondly, it can help identify areas which the students 
do not understand, and thus provide an implicit form of feedback. 
I believe that the endgaement in the classroom should be done in a fair, 
inclusive way, and in way such that a subset of students do not dominate 
the conversation.

  \js{get students to come to the board}
  
\customtitle{Fostering a sense of accomplishment:}
 %%%%%%%%%  
  I believe that should 
feel an accomplishment from learning. Even if they do not do so well 
in assessments, I believe evey student should leave with a net-positive 
outcome of any course. One way that I believe this can done is by 
engaging the students in small bits of active research, through either 
home-work assignments or more long-form course work. This should, of course, 
be designed at a level appropriate level.


  %%%%%%%%%  
\customtitle{Varied presentation styles}
 %%%%%%%%%  
 
 As someone with dyslexia, I understand
 the consequences of sitting in lectures  where the presentation 
style did not meet my needs. Thus I believe it is important that a varied presentation 
style is used for each and every bit of information given to the students. 
This sometimes will involve retracing material more than once, however this 
is not necessary a bad thing, as it can be used as reinforcement opportunities. 

 
%%%%%%%%%  
\customtitle{DEI}
%%%%%%%%% 
\begin{itemize}
\item I have a commitment promoting diversity, equality and inclusivity 
in the classroom and academia in general. 
\item I have profound dyslexia, and part of my contribution to DEI 
	had being 
\item To this end I have engaged with the arXiv team about acsessiblity 
in reasearch. 
\item I have also strived to make my reasearch and teach as accessible as possible
for example, making short video summarise available of written documents. 
\item During my time at OxHOS (discussed above) there was also a focused 
specifically on students from socio-economically disadvantaged backgrounds. 
\item Going forward. I will pan to follow the following targeted strategies: 
\end{itemize} 


%%%%%%%%%
%\begingroup
%\renewcommand{\section}[2]{}%
%%\renewcommand{\chapter}[2]{}% for other classes
%\begin{thebibliography}{}
%\bibitem{Davighi_Gripaios_Tooby-Smith_2020}
%    J. Davighi, B. Gripaios, and J. Tooby-Smith, J. Phys. A: Math. Theor. 53, 145302 (2020).
%\bibitem{U1Anomaly}
%    B. C. Allanach, B. Gripaios, and J. Tooby-Smith, J. High Energ. Phys. 2020, 65 (2020).
%\bibitem{GaugeRank}
%B. C. Allanach, B. Gripaios, and J. Tooby-Smith, Phys. Rev. D 101, 075015 (2020).
%\bibitem{ExtraGaugeBoson}
%B. C. Allanach, B. Gripaios, and J. Tooby-Smith, ArXiv:2006.03588 (2020).
%\bibitem{SuperSoft}
%T. Cohen, N. Craig, S. Koren, M. McCullough, and J. Tooby-Smith, ArXiv:2002.12630 (2020).
%\end{thebibliography}
%\endgroup

%%%%%%%%%%%%%%
\end{document}
