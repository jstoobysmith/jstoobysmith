\documentclass[14pt,letter]{article}
%%%
%Bool
\newcommand{\includePublications}{T} %T for true.
%%%
\usepackage[margin=0.5in]{geometry}
\usepackage{hyperref}
\renewcommand{\arraystretch}{1.5} 
\usepackage{enumitem}
\usepackage{xcolor}
%\definecolor{mycolor}{RGB}{37,129,120}
%\definecolor{mycolor}{RGB}{128, 0, 32}
\definecolor{mycolor}{RGB}{0, 128, 128}
%%%%%%%%%%%%%
%Title and entry
\usepackage{colortbl}
\newcommand{\mytitle}[1]{
   \setlength\arrayrulewidth{1.5pt}
	\noindent\begin{tabular}{p{0.9\textwidth}p{0.05\textwidth}}
	{\Large\bf #1}&{\hfill \large }\\
	\arrayrulecolor{mycolor}\hline
	\end{tabular}
}
\pagestyle{empty} 

\newcommand{\myentry}[2]{
	\noindent\begin{tabular}{p{0.1\textwidth}p{0.85\textwidth}}
	\textcolor{mycolor}{#1}& #2
	\end{tabular}\\
}
\usepackage{textcomp}
\newcommand{\mybullet}{\textcolor{mycolor}{$\ast$}\ }
%%%%%
\usepackage{titlesec}
\titleclass{\customtitle}{straight}[\section]
\newcounter{customtitle}
\titleformat{\customtitle}[block]
  {\normalfont}
  {}
  {0em}
  {\mytitle}
\titlespacing*{\customtitle}{-0.2cm}{*0.5}{0.1cm}
%%%%%%%%%%%%%
%For colored bar
\usepackage{eso-pic}


\AddToShipoutPictureBG*{%
 \AtPageUpperLeft{%
 \textcolor{mycolor}{\rule[-2.6cm]{\paperwidth}{4cm}}}}


\AddToShipoutPictureBG{%
 \AtPageUpperLeft{%
 \textcolor{mycolor}{\rule[-1cm]{\paperwidth}{4cm}}}}


\AddToShipoutPictureBG{%
 \AtPageLowerLeft{%
 \textcolor{mycolor}{\rule[-1.2cm]{\paperwidth}{2cm}}}}
 
\usepackage{tabularx}
%%%%%%%%%%%%%
\usepackage{draftwatermark}
\SetWatermarkText{\color{mycolor} DRAFT}
\SetWatermarkScale{9}

\begin{document}

%\center{(See separate document for publication list)}
\noindent{{\Huge \color{white}\bf Research Plan}\newline  }
\vspace{-1cm}
\begin{flushright}
{{\Large \color{white}Joseph Tooby-Smith }}
\end{flushright}
\vspace{0.4cm}
%%%%%%%%%  
\customtitle{Past research}
%%%%%%%%%  
The underlying theme of my research has been 
the application of techniques in pure mathematics and computer science 
to problems in the physicical sciences. This has lead to an 
expertise in two areas related to computer science: 
interactive theorem proving and category theory.
Let me dicuss these in turn.

\paragraph{Interactive theorem proving:} The main paper which demonstrates my skills in this area 
is \js{...}. This presents a program to digitalise results (meaning definitons, theorems and calculations)
from high energy physics into 
the interactive theorem prover Lean 4. This is the first anything like this has being attempted
in high energy physics. There a four important motivations to of project: \js{...}

Despite the application of this work been physics, the main challange of this project is 
use the correct tools in from computer science, and in particular functional programming.
To make the digitilisation as easy as possible. One such tool is the use of monads and operads
from category theory. This brings me onto my next area of expertise.

\pagraph{Category theory:} I have a strong background in the application of category theory 
outside the ivory towers of the pure mathematicians. Historically, my main use of category theory 
is as a language to recast problems from the physical sciences and to use this language 
to derive new previously unkown results. As a specific example, in high energy physics 
there is a relatively new notion called a "generalised symmetry", in  \js{...}, 
we used special types of categories called higher topoi to define and derive new results 
about these symmetries. 

Higher topos theory itself is related to homotopy type theory, which is actually 
the path that lead me to interactive theorem provers. 




On top of these I have worked on problems using the theory of Lie algebras,
combinatorics .... 


%%%%%%%%%  
\customtitle{Main future project: Theorem proving and and AI in the physical sciences}
%%%%%%%%%  

Going forward my main research goal is help progress interactive theorem provers, 
specfically Lean,
so that they can be used more easily in the physical sciences. 
In addition, I wish to work to further convince academics in the physical sciences 
that interactive theorem provers are a way forward in academic reasearch, and 
help build the bridge between the physical science, computer scientists 
working on interactive theorem provers, and those working on 
the use of AI in mathematics. 

To achieve this goal I have the following rough plan:

\begin{itemize}
\item Make index notation easy to use in HepLean.
\item Work towards a data set of thereoms in physics which can used 
to test AI Models. 
\item Create a blueprint for a physical science theory.
\item  Develop computational tactics to work with tensors
and their simplification. 
\item Make it easier for someone to jump in and contribute 
to HepLean.
\item Develop the general theory of feynman diagrams. 
\end{itemize}

\paragraph{Plan for undergradute and graduate involvement:} Part of the paper \js{...}, discusses the use of pedogoical use 
of HepLean.   

%%%%%%%%%  
\customtitle{Other future project: Higher category theory in computer science}
%%%%%%%%%  

\begin{itemize}
\item The reader may be familar with the notion of a monad used in functional programming.
\item Monads are a special case of a much more general notion in the theory of higher algebras. 
\item It is my believe that the application of more general notations of moads from higher algebra to computer science 
are little explored. 
\item My expertise in this area from past projects such as \js{..}, as well as my expertise in functional programming
 put me in a good position to explore this avenue.
\end{itemize}


%%%%%%%%%
%\begingroup
%\renewcommand{\section}[2]{}%
%%\renewcommand{\chapter}[2]{}% for other classes
%\begin{thebibliography}{}
%\bibitem{Davighi_Gripaios_Tooby-Smith_2020}
%    J. Davighi, B. Gripaios, and J. Tooby-Smith, J. Phys. A: Math. Theor. 53, 145302 (2020).
%\bibitem{U1Anomaly}
%    B. C. Allanach, B. Gripaios, and J. Tooby-Smith, J. High Energ. Phys. 2020, 65 (2020).
%\bibitem{GaugeRank}
%B. C. Allanach, B. Gripaios, and J. Tooby-Smith, Phys. Rev. D 101, 075015 (2020).
%\bibitem{ExtraGaugeBoson}
%B. C. Allanach, B. Gripaios, and J. Tooby-Smith, ArXiv:2006.03588 (2020).
%\bibitem{SuperSoft}
%T. Cohen, N. Craig, S. Koren, M. McCullough, and J. Tooby-Smith, ArXiv:2002.12630 (2020).
%\end{thebibliography}
%\endgroup

%%%%%%%%%%%%%%
\end{document}
