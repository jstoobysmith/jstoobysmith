\documentclass[14pt,letter]{article}
%%%
%Bool
\newcommand{\includePublications}{T} %T for true.
%%%
\usepackage[margin=0.5in]{geometry}
\usepackage{hyperref}
\renewcommand{\arraystretch}{1.5} 
\usepackage{enumitem}
\usepackage{xcolor}
%\definecolor{mycolor}{RGB}{37,129,120}
%\definecolor{mycolor}{RGB}{128, 0, 32}
\definecolor{mycolor}{RGB}{0, 128, 128}
%%%%%%%%%%%%%
%Title and entry
\usepackage{colortbl}
\newcommand{\mytitle}[1]{
   \setlength\arrayrulewidth{1.5pt}
	\noindent\begin{tabular}{p{0.9\textwidth}p{0.05\textwidth}}
	{\Large\bf #1}&{\hfill \large }\\
	\arrayrulecolor{mycolor}\hline
	\end{tabular}
}
\pagestyle{empty} 

\newcommand{\myentry}[2]{
	\noindent\begin{tabular}{p{0.1\textwidth}p{0.85\textwidth}}
	\textcolor{mycolor}{#1}& #2
	\end{tabular}\\
}
\usepackage{textcomp}
\newcommand{\mybullet}{\textcolor{mycolor}{$\ast$}\ }
%%%%%
\usepackage{titlesec}
\titleclass{\customtitle}{straight}[\section]
\newcounter{customtitle}
\titleformat{\customtitle}[block]
  {\normalfont}
  {}
  {0em}
  {\mytitle}
\titlespacing*{\customtitle}{-0.2cm}{*0.5}{0.1cm}
%%%%%%%%%%%%%
%For colored bar
\usepackage{eso-pic}


\AddToShipoutPictureBG*{%
 \AtPageUpperLeft{%
 \textcolor{mycolor}{\rule[-2.6cm]{\paperwidth}{4cm}}}}


\AddToShipoutPictureBG{%
 \AtPageUpperLeft{%
 \textcolor{mycolor}{\rule[-1cm]{\paperwidth}{4cm}}}}


\AddToShipoutPictureBG{%
 \AtPageLowerLeft{%
 \textcolor{mycolor}{\rule[-1.2cm]{\paperwidth}{2cm}}}}
 
\usepackage{tabularx}
%%%%%%%%%%%%%
\usepackage{draftwatermark}
\SetWatermarkText{\color{mycolor} DRAFT}
\SetWatermarkScale{9}

\begin{document}

%\center{(See separate document for publication list)}
\noindent{{\Huge \color{white}\bf Teaching philosophy}\newline  }
\vspace{-1cm}
\begin{flushright}
{{\Large \color{white}Joseph Tooby-Smith }}
\end{flushright}
\vspace{0.4cm}
%%%%%%%%%  
\customtitle{My teaching experience}
%%%%%%%%%  
\vspace{0.3cm}
\paragraph{High-school level students:} 
\begin{itemize}
\item During my undergraduate I undertook a "teaching in schools" program. 
\item Inovled, over the cause of a term, observing a teacher teach 
the physical science, and reasearch how misconseptions can arise in the 
class room.
\end{itemize}

\paragraph{Undergraduate and master students:} 

\paragraph{Doctoral students:} 

\paragraph{The general public:} 

%%%%%%%%%  
\customtitle{EDI}
%%%%%%%%% 
\begin{itemize}
\item I have profound dyslexia and this experience has taught me the importance 
	of making teaching and research inclusive.
\item To this end I have engaged with the arXiv team about acsessiblity 
in reasearch. 
\item I have also strived to make my reasearch and teach as accessible as possible
for example, making short video summarise available of written documents. 
\item During my time at OxHOS (discussed above) there was also a focused 
specifically on students from socio-economically disadvantaged backgrounds. 
\item Going forward. I will pan to follow the following targeted strategies: 
\end{itemize} 

%%%%%%%%%  
\customtitle{Contribution to the pedology literature}
%%%%%%%%%  
\begin{itemize}
\item  As discussed in my research proposal, I have worked developed 
the project HepLean.
\item One of the main motivating reasons for HepLean was pedagogy. 
\item In the associated paper it was discussed how a project like HepLean can 
help teach both the physical sciences and computer science.
\end{itemize}
%%%%%%%%%  
\customtitle{My Teaching Philosophy}
%%%%%%%%%  
In this section I will discuss the specifics of my teaching philosophy, 
and more importantly, how I implement that teaching philosophy. 
It is important to note that I don't claim any of methods here are the correct 
ones, as I am always on the look out for better teaching methods.

My teaching philosophy can be broken down into to five statements.

\paragraph{Tailored, level-appropriate material:} Having worked at the boundary of three different disciplines (physics, mathematics and computer science), and having presented on many occasions material from one  discipline to academics in others, I am aware of the importance of presenting material at the correct level for the audience. I'm also aware of what goes wrong when this is not done. 

\paragraph{Active listening to feedback:} e.g. Anonymous questions.

\paragraph{Student-centered engagement:}  e.g. self-explaination

\paragraph{Fostering a sense of accomplishment:} e.g. contributing to an active bit of research, even in a small way. 

\paragraph{Varied presentation styles:} As someone 

%%%%%%%%%
%\begingroup
%\renewcommand{\section}[2]{}%
%%\renewcommand{\chapter}[2]{}% for other classes
%\begin{thebibliography}{}
%\bibitem{Davighi_Gripaios_Tooby-Smith_2020}
%    J. Davighi, B. Gripaios, and J. Tooby-Smith, J. Phys. A: Math. Theor. 53, 145302 (2020).
%\bibitem{U1Anomaly}
%    B. C. Allanach, B. Gripaios, and J. Tooby-Smith, J. High Energ. Phys. 2020, 65 (2020).
%\bibitem{GaugeRank}
%B. C. Allanach, B. Gripaios, and J. Tooby-Smith, Phys. Rev. D 101, 075015 (2020).
%\bibitem{ExtraGaugeBoson}
%B. C. Allanach, B. Gripaios, and J. Tooby-Smith, ArXiv:2006.03588 (2020).
%\bibitem{SuperSoft}
%T. Cohen, N. Craig, S. Koren, M. McCullough, and J. Tooby-Smith, ArXiv:2002.12630 (2020).
%\end{thebibliography}
%\endgroup

%%%%%%%%%%%%%%
\end{document}
