\documentclass[14pt,letter]{article}
%%%
%Bool
\newcommand{\includePublications}{T} %T for true.
%%%
\usepackage[margin=0.5in]{geometry}
\usepackage{hyperref}
\renewcommand{\arraystretch}{1.5} 
\usepackage{enumitem}
\usepackage{xcolor}
%\definecolor{mycolor}{RGB}{37,129,120}
%\definecolor{mycolor}{RGB}{128, 0, 32}
\definecolor{mycolor}{RGB}{0, 128, 128}
%%%%%%%%%%%%%
%Title and entry
\usepackage{colortbl}
\newcommand{\mytitle}[1]{
   \setlength\arrayrulewidth{1.5pt}
	\noindent\begin{tabular}{p{0.9\textwidth}p{0.05\textwidth}}
	{\Large\bf #1}&{\hfill \large }\\
	\arrayrulecolor{mycolor}\hline
	\end{tabular}
}
\pagestyle{empty} 

\newcommand{\myentry}[2]{
	\noindent\begin{tabular}{p{0.1\textwidth}p{0.85\textwidth}}
	\textcolor{mycolor}{#1}& #2
	\end{tabular}\\
}
\usepackage{textcomp}
\newcommand{\mybullet}{\textcolor{mycolor}{$\ast$}\ }
%%%%%
\usepackage{titlesec}
\titleclass{\customtitle}{straight}[\section]
\newcounter{customtitle}
\titleformat{\customtitle}[block]
  {\normalfont}
  {}
  {0em}
  {\mytitle}
\titlespacing*{\customtitle}{-0.2cm}{*0.5}{0.1cm}
%%%%%%%%%%%%%
%For colored bar
\usepackage{eso-pic}


\AddToShipoutPictureBG*{%
 \AtPageUpperLeft{%
 \textcolor{mycolor}{\rule[-2.6cm]{\paperwidth}{4cm}}}}


\AddToShipoutPictureBG{%
 \AtPageUpperLeft{%
 \textcolor{mycolor}{\rule[-1cm]{\paperwidth}{4cm}}}}


\AddToShipoutPictureBG{%
 \AtPageLowerLeft{%
 \textcolor{mycolor}{\rule[-1.2cm]{\paperwidth}{2cm}}}}
 
\usepackage{tabularx}
%%%%%%%%%%%%%
\usepackage{draftwatermark}
\SetWatermarkText{\color{mycolor} DRAFT}
\SetWatermarkScale{9}

%% Comments 

\begin{document}

%\center{(See separate document for publication list)}
\noindent{{\Huge \color{white}\bf Research Plan}\newline  }
\vspace{-1cm}
\begin{flushright}
{{\Large \color{white}Joseph Tooby-Smith }}
\end{flushright}
\vspace{0.4cm}

My research sits at the intersection of computer science, 
mathematics and physics.
I am intrested in the building of a bridge between these areas using 
interactive theorem proving. I have PhD in theoretical physics 
from the University of Cambridge, 
have completed a postdoc at Cornell University in which I focused on the 
application of theorem proving software in Physics. For the academic 
year 2024-2025 I am undertaking a postdoc in computer science at the 
University of Reykjavik.
%%%%%%%%%  
\customtitle{The why}
%%%%%%%%%  

I have a background in high energy physics. Looking forward to the future of high energy physics, and the physical sciences more generally, there are a number of things that I would like to see: 

\begin{enumerate}
\item A linear-storage of information:
\item Automated methods to prove new results:
\item Automatic review of results for mathematical correctness: 
\item New pedagogy methods:	
\end{enumerate}

%%%%%%%%%  
\customtitle{The past}
%%%%%%%%%  

One tool we can use to achieve the goals mentioned above is use interactive theorem provers. Part of my past research has been in this direction, developing a libary called `HepLean'. The goal of HepLean is to digitalise results 
(meaning definitons, theorems and calculations)
from high energy physics into 
the interactive theorem prover Lean 4. This is the first anything like this has being attempted
in high energy physics. However, there is a similar project for mathematics called Mathlib, which forms the mathematical foundation underlying HepLean.

Currently HepLean contains a number of results from high-energy physics, and one future aim is to expand its scope. Let me discuss how HepLean solves the four problems above: 

\begin{enumerate}
\item A linear-storage of information:
\item Automated methods to prove new results:
\item Automatic review of results for mathematical correctness: 
\item New pedagogy methods:	
\end{enumerate}

%%%%%%%%%  
\customtitle{The future}
%%%%%%%%%  
My plan for future research is to continue the development of HepLean, with a focus on the theoretical and computer science foundations, to make it easier for the average physical scientist to adapt to using Lean. In practice this will mean applying techniques from functional programming, AI, category theory etc. to write foundational definitions in the most effective and efficient way. 

To acheive this overarching goal, I plan to undertake the following specific research steps:  
\begin{enumerate}
\item In Lean 4 there is notion of blueprint for a theory. This is 
a English-written document containg all of the steps that must be taken
to turn the prove of an English-written prove into a Lean written prove. 
This can be thought of as pseudo-code for Lean. To help build the above 
bridge I would produce such a pseudo-code for an theory in physics.
\item Most work on AI in mathematics has looked at e.g. math Olympiad problems
in Lean (e.g. Google Deepminds work). I would like to see the use 
of AI to solve problems from the physical science in Lean. To do this 
I plan to create a data set of Lean 4 written theorems from physics 
that can be used for AI testing and training.
\item Overlapping a bit with AI, high energy physics use heavily tensors. 
As part of Lean 4 I would like to develop tactics that help formally verify 
results related to tensors.
\end{enumerate}

\customtitle{Plan for undergraduate student involvement} 


Part of the paper \js{...}, discusses how HepLean can be used 
as a pedagogical tool, and give students the ability to get involved in research.  My plan is to develop three list of undergraduate-level projects around HepLean. 

The first list will be concerned with functional programming type projects. As an example: the handling of lists in Lean to efficiently undertake computations needed for index notation (a notation used by physicists to deal with tensors). Other examples will involve meta-programming in Lean to make the user-experience easier.

The second list of projects  will be concerned with the use of AI for physics and mathematics. Simple example involve auto-formalisation of theorems in physics (turning a human written result into a result written in Lean), as will as the inverse problem, `auto-informalisation'. These have being heavily explored in mathematics, but not in the physical sciences. 

The third list will be at the boundary of physics, computer-science and mathematics. These projects will involve a proving theorems from physics using Lean. There are many such problems that involve very little prerequiests in physics, once the theorem has being written down. Part of my plan above, with the blueprint, will be a first step in this direction. An example of such a project will be to formalisation of properties of the two-Higgs doublet model potential. 
This is a potential, and physicists are interested in its properties, 
such as its minima, whether it is bounded or not etc.

Each of these lists of projects will involve `home-work style projects' which will take no more than a couple of hours to complete, and more detailed thesis level projects. The benefit of having a large project like HepLean is that coming up with such a diversity of projects is relatively easy. 

%%%%%%%%%  
\customtitle{Other research}
%%%%%%%%%  
 I have a strong background in the application of category theory 
outside the ivory towers of the pure mathematicians. Historically, my main use of category theory 
is as a language to recast problems from the physical sciences and to use this language 
to derive new previously unkown results. As a specific example, in high energy physics 
there is a relatively new notion called a "generalised symmetry", in  \js{...}, 
we used special types of categories called higher topoi to define and derive new results 
about these symmetries. Higher topos theory itself is related to homotopy type theory, which is actually 
the path that lead me to interactive theorem provers. 

Category theory is used heaivly in functional programming, for example, the definition a monad is categorical-definition. It turns out that a moad 
a special case of a more general object in the theory of Higher algebra. 
This is an area I have expertise, since it overlaps with my use of category theory in physics, as demonstrated in \emph{e.g.} \js{}. 

In the future I also plan to investigate the role (higher) category theory can play in functional programming and more specifically interactive theorem provers for the physical science. I'm currently making some investigations into area, using the theory of modular operads to develop an efficient method for index notation of tensors. 

%%%%%%%%%
%\begingroup
%\renewcommand{\section}[2]{}%
%%\renewcommand{\chapter}[2]{}% for other classes
%\begin{thebibliography}{}
%\bibitem{Davighi_Gripaios_Tooby-Smith_2020}
%    J. Davighi, B. Gripaios, and J. Tooby-Smith, J. Phys. A: Math. Theor. 53, 145302 (2020).
%\bibitem{U1Anomaly}
%    B. C. Allanach, B. Gripaios, and J. Tooby-Smith, J. High Energ. Phys. 2020, 65 (2020).
%\bibitem{GaugeRank}
%B. C. Allanach, B. Gripaios, and J. Tooby-Smith, Phys. Rev. D 101, 075015 (2020).
%\bibitem{ExtraGaugeBoson}
%B. C. Allanach, B. Gripaios, and J. Tooby-Smith, ArXiv:2006.03588 (2020).
%\bibitem{SuperSoft}
%T. Cohen, N. Craig, S. Koren, M. McCullough, and J. Tooby-Smith, ArXiv:2002.12630 (2020).
%\end{thebibliography}
%\endgroup

%%%%%%%%%%%%%%
\end{document}
