\documentclass[14pt,letter]{article}
%%%
%Bool
\newcommand{\includePublications}{T} %T for true.
%%%
\usepackage[margin=0.5in]{geometry}
\usepackage{hyperref}
\renewcommand{\arraystretch}{1.5} 
\usepackage{enumitem}
\usepackage{xcolor}
%\definecolor{mycolor}{RGB}{37,129,120}
%\definecolor{mycolor}{RGB}{128, 0, 32}
\definecolor{mycolor}{RGB}{0, 128, 128}
%%%%%%%%%%%%%
%Title and entry
\usepackage{colortbl}
\newcommand{\mytitle}[1]{
   \setlength\arrayrulewidth{1.5pt}
	\noindent\begin{tabular}{p{0.9\textwidth}p{0.05\textwidth}}
	{\Large\bf #1}&{\hfill \large }\\
	\arrayrulecolor{mycolor}\hline
	\end{tabular}
}
\pagestyle{empty} 

\newcommand{\myentry}[2]{
	\noindent\begin{tabular}{p{0.1\textwidth}p{0.85\textwidth}}
	\textcolor{mycolor}{#1}& #2
	\end{tabular}\\
}
\usepackage{textcomp}
\newcommand{\mybullet}{\textcolor{mycolor}{$\ast$}\ }
%%%%%
\usepackage{titlesec}
\titleclass{\customtitle}{straight}[\section]
\newcounter{customtitle}
\titleformat{\customtitle}[block]
  {\normalfont}
  {}
  {0em}
  {\mytitle}
\titlespacing*{\customtitle}{-0.2cm}{*0.5}{0.1cm}
%%%%%%%%%%%%%
%For colored bar
\usepackage{eso-pic}


\AddToShipoutPictureBG*{%
 \AtPageUpperLeft{%
 \textcolor{mycolor}{\rule[-2.6cm]{\paperwidth}{4cm}}}}


\AddToShipoutPictureBG{%
 \AtPageUpperLeft{%
 \textcolor{mycolor}{\rule[-1cm]{\paperwidth}{4cm}}}}


\AddToShipoutPictureBG{%
 \AtPageLowerLeft{%
 \textcolor{mycolor}{\rule[-1.2cm]{\paperwidth}{2cm}}}}
 
\usepackage{tabularx}
%%%%%%%%%%%%%
\usepackage{draftwatermark}
\SetWatermarkText{\color{mycolor} DRAFT}
\SetWatermarkScale{9}

\begin{document}

%\center{(See separate document for publication list)}
\noindent{{\Huge \color{white}\bf Research Plan}\newline  }
\vspace{-1cm}
\begin{flushright}
{{\Large \color{white}Joseph Tooby-Smith }}
\end{flushright}
\vspace{0.4cm}
%%%%%%%%%  
\customtitle{Past research}
%%%%%%%%%  
The underlying theme of my research has been 
the application of techniques in pure mathematics and computer science 
to problems in the physicical sciences. This has lead to an 
expertise in two areas related to computer science: 
interactive theorem proving and category theory.
Let me dicuss these in turn.

\paragraph{Interactive theorem proving:}

\pagraph{Category theory:}


On top of these I have worked on problems using the theory of Lie algebras,
combinatorics .... 


%%%%%%%%%  
\customtitle{Main goal: Theorem proving and and AI in the physical sciences}
%%%%%%%%%  

The main goal of my future research is develop HepLean, and the more 
general use of interactive theorem provers in the physical sciences. 

\begin{itemize}
\item Make index notation easy to use in HepLean.
\item Work towards a data set of thereoms in physics which can used 
to test AI Models. 
\item Create a blueprint for a physical science theory.
\item  Develop computational tactics to work with tensors
and their simplification. 
\item Make it easier for someone to jump in and contribute 
to HepLean.
\item Develop the general theory of feynman diagrams. 
\end{itemize}

\paragraph{Plan for undergradute involvement:} Part of the paper \js{...}, discusses the use of pedogoical use 
of HepLean.   
%%%%%%%%%  
\customtitle{Category theory in computer science}
%%%%%%%%%  

\begin{itemize}
\item My background from physics and mathematics have given me 
knowledge of category theory.
\item Modular operads. 
\item Higher algebras. 
\item Groups etc.
\end{itemize}

%%%%%%%%%
%\begingroup
%\renewcommand{\section}[2]{}%
%%\renewcommand{\chapter}[2]{}% for other classes
%\begin{thebibliography}{}
%\bibitem{Davighi_Gripaios_Tooby-Smith_2020}
%    J. Davighi, B. Gripaios, and J. Tooby-Smith, J. Phys. A: Math. Theor. 53, 145302 (2020).
%\bibitem{U1Anomaly}
%    B. C. Allanach, B. Gripaios, and J. Tooby-Smith, J. High Energ. Phys. 2020, 65 (2020).
%\bibitem{GaugeRank}
%B. C. Allanach, B. Gripaios, and J. Tooby-Smith, Phys. Rev. D 101, 075015 (2020).
%\bibitem{ExtraGaugeBoson}
%B. C. Allanach, B. Gripaios, and J. Tooby-Smith, ArXiv:2006.03588 (2020).
%\bibitem{SuperSoft}
%T. Cohen, N. Craig, S. Koren, M. McCullough, and J. Tooby-Smith, ArXiv:2002.12630 (2020).
%\end{thebibliography}
%\endgroup

%%%%%%%%%%%%%%
\end{document}
